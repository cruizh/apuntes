\documentclass[spanish,]{book}
\usepackage{lmodern}
\usepackage{amssymb,amsmath}
\usepackage{ifxetex,ifluatex}
\usepackage{fixltx2e} % provides \textsubscript
\ifnum 0\ifxetex 1\fi\ifluatex 1\fi=0 % if pdftex
  \usepackage[T1]{fontenc}
  \usepackage[utf8]{inputenc}
\else % if luatex or xelatex
  \ifxetex
    \usepackage{mathspec}
  \else
    \usepackage{fontspec}
  \fi
  \defaultfontfeatures{Ligatures=TeX,Scale=MatchLowercase}
\fi
% use upquote if available, for straight quotes in verbatim environments
\IfFileExists{upquote.sty}{\usepackage{upquote}}{}
% use microtype if available
\IfFileExists{microtype.sty}{%
\usepackage{microtype}
\UseMicrotypeSet[protrusion]{basicmath} % disable protrusion for tt fonts
}{}
\usepackage[margin=1in]{geometry}
\usepackage{hyperref}
\hypersetup{unicode=true,
            pdftitle={Teoría de la probabilidad},
            pdfauthor={Carlos José Ruiz-Henestrosa Ruiz},
            pdfborder={0 0 0},
            breaklinks=true}
\urlstyle{same}  % don't use monospace font for urls
\ifnum 0\ifxetex 1\fi\ifluatex 1\fi=0 % if pdftex
  \usepackage[shorthands=off,main=spanish]{babel}
\else
  \usepackage{polyglossia}
  \setmainlanguage[]{spanish}
\fi
\usepackage{natbib}
\bibliographystyle{apalike}
\usepackage{color}
\usepackage{fancyvrb}
\newcommand{\VerbBar}{|}
\newcommand{\VERB}{\Verb[commandchars=\\\{\}]}
\DefineVerbatimEnvironment{Highlighting}{Verbatim}{commandchars=\\\{\}}
% Add ',fontsize=\small' for more characters per line
\usepackage{framed}
\definecolor{shadecolor}{RGB}{248,248,248}
\newenvironment{Shaded}{\begin{snugshade}}{\end{snugshade}}
\newcommand{\AlertTok}[1]{\textcolor[rgb]{0.94,0.16,0.16}{#1}}
\newcommand{\AnnotationTok}[1]{\textcolor[rgb]{0.56,0.35,0.01}{\textbf{\textit{#1}}}}
\newcommand{\AttributeTok}[1]{\textcolor[rgb]{0.77,0.63,0.00}{#1}}
\newcommand{\BaseNTok}[1]{\textcolor[rgb]{0.00,0.00,0.81}{#1}}
\newcommand{\BuiltInTok}[1]{#1}
\newcommand{\CharTok}[1]{\textcolor[rgb]{0.31,0.60,0.02}{#1}}
\newcommand{\CommentTok}[1]{\textcolor[rgb]{0.56,0.35,0.01}{\textit{#1}}}
\newcommand{\CommentVarTok}[1]{\textcolor[rgb]{0.56,0.35,0.01}{\textbf{\textit{#1}}}}
\newcommand{\ConstantTok}[1]{\textcolor[rgb]{0.00,0.00,0.00}{#1}}
\newcommand{\ControlFlowTok}[1]{\textcolor[rgb]{0.13,0.29,0.53}{\textbf{#1}}}
\newcommand{\DataTypeTok}[1]{\textcolor[rgb]{0.13,0.29,0.53}{#1}}
\newcommand{\DecValTok}[1]{\textcolor[rgb]{0.00,0.00,0.81}{#1}}
\newcommand{\DocumentationTok}[1]{\textcolor[rgb]{0.56,0.35,0.01}{\textbf{\textit{#1}}}}
\newcommand{\ErrorTok}[1]{\textcolor[rgb]{0.64,0.00,0.00}{\textbf{#1}}}
\newcommand{\ExtensionTok}[1]{#1}
\newcommand{\FloatTok}[1]{\textcolor[rgb]{0.00,0.00,0.81}{#1}}
\newcommand{\FunctionTok}[1]{\textcolor[rgb]{0.00,0.00,0.00}{#1}}
\newcommand{\ImportTok}[1]{#1}
\newcommand{\InformationTok}[1]{\textcolor[rgb]{0.56,0.35,0.01}{\textbf{\textit{#1}}}}
\newcommand{\KeywordTok}[1]{\textcolor[rgb]{0.13,0.29,0.53}{\textbf{#1}}}
\newcommand{\NormalTok}[1]{#1}
\newcommand{\OperatorTok}[1]{\textcolor[rgb]{0.81,0.36,0.00}{\textbf{#1}}}
\newcommand{\OtherTok}[1]{\textcolor[rgb]{0.56,0.35,0.01}{#1}}
\newcommand{\PreprocessorTok}[1]{\textcolor[rgb]{0.56,0.35,0.01}{\textit{#1}}}
\newcommand{\RegionMarkerTok}[1]{#1}
\newcommand{\SpecialCharTok}[1]{\textcolor[rgb]{0.00,0.00,0.00}{#1}}
\newcommand{\SpecialStringTok}[1]{\textcolor[rgb]{0.31,0.60,0.02}{#1}}
\newcommand{\StringTok}[1]{\textcolor[rgb]{0.31,0.60,0.02}{#1}}
\newcommand{\VariableTok}[1]{\textcolor[rgb]{0.00,0.00,0.00}{#1}}
\newcommand{\VerbatimStringTok}[1]{\textcolor[rgb]{0.31,0.60,0.02}{#1}}
\newcommand{\WarningTok}[1]{\textcolor[rgb]{0.56,0.35,0.01}{\textbf{\textit{#1}}}}
\usepackage{longtable,booktabs}
\usepackage{graphicx,grffile}
\makeatletter
\def\maxwidth{\ifdim\Gin@nat@width>\linewidth\linewidth\else\Gin@nat@width\fi}
\def\maxheight{\ifdim\Gin@nat@height>\textheight\textheight\else\Gin@nat@height\fi}
\makeatother
% Scale images if necessary, so that they will not overflow the page
% margins by default, and it is still possible to overwrite the defaults
% using explicit options in \includegraphics[width, height, ...]{}
\setkeys{Gin}{width=\maxwidth,height=\maxheight,keepaspectratio}
\IfFileExists{parskip.sty}{%
\usepackage{parskip}
}{% else
\setlength{\parindent}{0pt}
\setlength{\parskip}{6pt plus 2pt minus 1pt}
}
\setlength{\emergencystretch}{3em}  % prevent overfull lines
\providecommand{\tightlist}{%
  \setlength{\itemsep}{0pt}\setlength{\parskip}{0pt}}
\setcounter{secnumdepth}{5}
% Redefines (sub)paragraphs to behave more like sections
\ifx\paragraph\undefined\else
\let\oldparagraph\paragraph
\renewcommand{\paragraph}[1]{\oldparagraph{#1}\mbox{}}
\fi
\ifx\subparagraph\undefined\else
\let\oldsubparagraph\subparagraph
\renewcommand{\subparagraph}[1]{\oldsubparagraph{#1}\mbox{}}
\fi

%%% Use protect on footnotes to avoid problems with footnotes in titles
\let\rmarkdownfootnote\footnote%
\def\footnote{\protect\rmarkdownfootnote}

%%% Change title format to be more compact
\usepackage{titling}

% Create subtitle command for use in maketitle
\newcommand{\subtitle}[1]{
  \posttitle{
    \begin{center}\large#1\end{center}
    }
}

\setlength{\droptitle}{-2em}
  \title{Teoría de la probabilidad}
  \pretitle{\vspace{\droptitle}\centering\huge}
  \posttitle{\par}
  \author{Carlos José Ruiz-Henestrosa Ruiz}
  \preauthor{\centering\large\emph}
  \postauthor{\par}
  \predate{\centering\large\emph}
  \postdate{\par}
  \date{2018-02-07}

\usepackage{booktabs}
\usepackage{amsthm}

\newcommand{\EV}[1]{\mathbb{E}\left[#1\right]}
\renewcommand{\P}[1]{\mathbb{P}\left[#1\right]}
\newcommand{Norm}[2]{\operatorname{N}\left(#1,#2\right)}

\makeatletter
\def\thm@space@setup{%
  \thm@preskip=8pt plus 2pt minus 4pt
  \thm@postskip=\thm@preskip
}
\makeatother

\begin{document}
\maketitle

{
\setcounter{tocdepth}{1}
\tableofcontents
}
\hypertarget{requisitos-previos}{%
\chapter*{Requisitos previos}\label{requisitos-previos}}
\addcontentsline{toc}{chapter}{Requisitos previos}

Este libro está escrito en \textbf{Markdown}, concretamente el dialecto
de \href{https://pandoc.org}{Pandoc}. Puedes consultar el código fuente
que ha generado este sitio web en
\href{https://github.com/cruizh/apuntes}{el repositorio de GitHub} o
pulsando el botón de edición de la barra superior.

La compilación de este libro se ha realizado mediante \textbf{bookdown}.
Se puede instalar desde CRAN o GitHub:

\begin{Shaded}
\begin{Highlighting}[]
\KeywordTok{install.packages}\NormalTok{(}\StringTok{"bookdown"}\NormalTok{)}
\CommentTok{# or the development version}
\CommentTok{# devtools::install_github("rstudio/bookdown")}
\end{Highlighting}
\end{Shaded}

\hypertarget{informacion-de-la-asignatura}{%
\chapter*{Información de la
asignatura}\label{informacion-de-la-asignatura}}
\addcontentsline{toc}{chapter}{Información de la asignatura}

\hypertarget{profesores}{%
\section{Profesores}\label{profesores}}

El profesor de la asignatura es Fernando López Blázquez. Las tutorías
son los lunes de 09:00 a 12:00 y miércoles y viernes de 11:00 a 12:30.
Su despacho se encuentra en el puente 23.

\hypertarget{bibliografia-recomendada}{%
\section{Bibliografía recomendada}\label{bibliografia-recomendada}}

\hypertarget{evaluacion-alternativa}{%
\section{Evaluación alternativa}\label{evaluacion-alternativa}}

La asignatura tendrá un examen prefinal en el que se deberán obtener 3
puntos como mínimo en la parte de teoría y la de problemas y un 5 de
media.

\hypertarget{vectores-aleatorios.-independencia.}{%
\chapter{Vectores aleatorios.
Independencia.}\label{vectores-aleatorios.-independencia.}}

\hypertarget{ejercicios}{%
\section{Ejercicios}\label{ejercicios}}

\hypertarget{ejercicio-1}{%
\subsection{Ejercicio 1}\label{ejercicio-1}}

Sabemos que \(X \sim \Norm{\mu}{\sigma^{2}}\). Entonces su función de
distribución es \(F_{X}(x) = \Phi \left(\frac{x-\mu}{\sigma}\right)\).
Entonces
\(\P{X > 7} = 0.5 \implies \P{X \leq 7} = \Phi \left(\frac{x-\mu}{\sigma}\right) = 1-0.5 = 0.5\).
Como la distribución es simétrica respecto de \(\mu\), entonces
\(\mu = 7\).

Por otra parte,
\(\P{X < 5.04} = \Phi \left(\frac{5.04-7}{\sigma} \right) = 0.025 \implies \frac{-1.96}{\sigma} = \operatorname{qnorm}(0.025) \approx -1.96\).
Por tanto, \(\sigma = 1\).

\begin{enumerate}
\def\labelenumi{\alph{enumi}.}
\tightlist
\item
  \[\P{X \leq 6} = \P{Z \leq \frac{6-7}{1}} = \Phi(-1) \approx 0.159\].
\item
  Llamamos \(C\) al suceso ``encontrar trabajo en 6 meses''. Entonces
  \[\P{C} = \Pr{C|A}\P{A} + \P{C|B}\Pr{B} = 0.4 \cdot \Phi(-1) + 0.6 \cdot 0.2 \approx 0.183\].
\end{enumerate}

\hypertarget{bibliografia}{%
\chapter*{Bibliografía}\label{bibliografia}}
\addcontentsline{toc}{chapter}{Bibliografía}

\bibliography{../book.bib,packages.bib}


\end{document}
